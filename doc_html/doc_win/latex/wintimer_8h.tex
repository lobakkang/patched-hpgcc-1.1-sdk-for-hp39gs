\section{wintimer.h File Reference}
\label{wintimer_8h}\index{wintimer.h@{wintimer.h}}
Functions related to setting and executing timers. 

\subsection*{Typedefs}
\begin{CompactItemize}
\item 
typedef int($\ast$ {\bf win\_\-timerfunc\_\-t} )(void $\ast$app\_\-data)
\begin{CompactList}\small\item\em Function signature for a timer callback. \item\end{CompactList}\end{CompactItemize}
\subsection*{Functions}
\begin{CompactItemize}
\item 
void {\bf win\_\-add\_\-timer} ({\bf win\_\-timerfunc\_\-t} func, void $\ast$app\_\-data, int interval\_\-ms, int periodic)
\begin{CompactList}\small\item\em Adds a timer to call a registered callback function. \item\end{CompactList}\item 
void {\bf win\_\-add\_\-timer\_\-event} (int event\_\-num, void $\ast$event\_\-data, int interval\_\-ms, int periodic)
\begin{CompactList}\small\item\em Adds a timer to generate events on the event queue. \item\end{CompactList}\item 
void {\bf win\_\-remove\_\-timer} ({\bf win\_\-timerfunc\_\-t} func, void $\ast$app\_\-data)
\begin{CompactList}\small\item\em Removes a callback-style timer. \item\end{CompactList}\item 
void {\bf win\_\-remove\_\-timer\_\-event} (int event\_\-num, void $\ast$event\_\-data)
\begin{CompactList}\small\item\em Removes an event-style timer. \item\end{CompactList}\end{CompactItemize}


\subsection{Detailed Description}
Functions related to setting and executing timers. 

Timers provide a way for an application using the HPGCC window system to perform periodic tasks, such as updating a clock, progressing to the next frame of an animation, or flashing a text cursor. 

Definition in file {\bf wintimer.h}.

\subsection{Typedef Documentation}
\index{wintimer.h@{wintimer.h}!win_timerfunc_t@{win\_\-timerfunc\_\-t}}
\index{win_timerfunc_t@{win\_\-timerfunc\_\-t}!wintimer.h@{wintimer.h}}
\subsubsection{\setlength{\rightskip}{0pt plus 5cm}typedef int($\ast$ {\bf win\_\-timerfunc\_\-t})(void $\ast$app\_\-data)}\label{wintimer_8h_a0}


Function signature for a timer callback. 

\begin{Desc}
\item[Parameters:]
\begin{description}
\item[{\em app\_\-data}]A data field with arbitrary contents. This value is passed to {\bf win\_\-add\_\-timer}{\rm (p.\,\pageref{wintimer_8h_a1})}, and is returned to the callback function when a timer fires.\end{description}
\end{Desc}
\begin{Desc}
\item[Returns:]Zero if the timer should continue firing; non-zero to drop the timer. \end{Desc}


Definition at line 55 of file wintimer.h.

\subsection{Function Documentation}
\index{wintimer.h@{wintimer.h}!win_add_timer@{win\_\-add\_\-timer}}
\index{win_add_timer@{win\_\-add\_\-timer}!wintimer.h@{wintimer.h}}
\subsubsection{\setlength{\rightskip}{0pt plus 5cm}void win\_\-add\_\-timer ({\bf win\_\-timerfunc\_\-t} {\em func}, void $\ast$ {\em app\_\-data}, int {\em interval\_\-ms}, int {\em periodic})}\label{wintimer_8h_a1}


Adds a timer to call a registered callback function. 

\begin{Desc}
\item[Parameters:]
\begin{description}
\item[{\em func}]Pointer to the timer callback function. \item[{\em app\_\-data}]An arbitrary data field that is kept, and passed to the callback function every time it is invoked. \item[{\em interval\_\-ms}]The interval between successive calls to the callback function, measured in milliseconds. \item[{\em periodic}]If non-zero, then {\tt interval\_\-ms} specifies the period of the repetitive action. If zero, then {\tt interval\_\-ms} is the delay between actions. The difference is that when {\tt periodic} is zero, certain error terms - such as the time taken to execute the callback, and any lateness of the callback being run, are cumulative. When {\tt periodic} is non-zero, the next delay is shortened to correct for that kind of error, so the error does not accumulate. \end{description}
\end{Desc}
\index{wintimer.h@{wintimer.h}!win_add_timer_event@{win\_\-add\_\-timer\_\-event}}
\index{win_add_timer_event@{win\_\-add\_\-timer\_\-event}!wintimer.h@{wintimer.h}}
\subsubsection{\setlength{\rightskip}{0pt plus 5cm}void win\_\-add\_\-timer\_\-event (int {\em event\_\-num}, void $\ast$ {\em event\_\-data}, int {\em interval\_\-ms}, int {\em periodic})}\label{wintimer_8h_a2}


Adds a timer to generate events on the event queue. 

\begin{Desc}
\item[Parameters:]
\begin{description}
\item[{\em event\_\-num}]The event number to generate when the timer fires. \item[{\em event\_\-data}]An arbitrary data field that is kept, and passed to the event when it is delivered. \item[{\em interval\_\-ms}]The interval between successive calls to the callback function, measured in milliseconds. \item[{\em periodic}]If non-zero, then {\tt interval\_\-ms} specifies the period of the repetitive action. If zero, then {\tt interval\_\-ms} is the delay between actions. The difference is that when {\tt periodic} is zero, certain error terms - such as the time taken to execute the callback, and any lateness of the callback being run, are cumulative. When {\tt periodic} is non-zero, the next delay is shortened to correct for that kind of error, so the error does not accumulate. \end{description}
\end{Desc}
\index{wintimer.h@{wintimer.h}!win_remove_timer@{win\_\-remove\_\-timer}}
\index{win_remove_timer@{win\_\-remove\_\-timer}!wintimer.h@{wintimer.h}}
\subsubsection{\setlength{\rightskip}{0pt plus 5cm}void win\_\-remove\_\-timer ({\bf win\_\-timerfunc\_\-t} {\em func}, void $\ast$ {\em app\_\-data})}\label{wintimer_8h_a3}


Removes a callback-style timer. 

There are two ways to remove a timer. One way is to call this function to remove it explicitly. The other is to return a non-zero value from the timer callback itself.

\begin{Desc}
\item[Parameters:]
\begin{description}
\item[{\em func}]The callback function to remove \item[{\em app\_\-data}]The data field for the callback to remove \end{description}
\end{Desc}
\index{wintimer.h@{wintimer.h}!win_remove_timer_event@{win\_\-remove\_\-timer\_\-event}}
\index{win_remove_timer_event@{win\_\-remove\_\-timer\_\-event}!wintimer.h@{wintimer.h}}
\subsubsection{\setlength{\rightskip}{0pt plus 5cm}void win\_\-remove\_\-timer\_\-event (int {\em event\_\-num}, void $\ast$ {\em event\_\-data})}\label{wintimer_8h_a4}


Removes an event-style timer. 

There are two ways to remove an event timer. One way is to call this function to remove it explicitly. The other is to return a non-zero value from an event handler registered for that event.

\begin{Desc}
\item[Parameters:]
\begin{description}
\item[{\em event\_\-num}]The event number of the event timer to remove \item[{\em event\_\-data}]The event data field for the event timer to remove \end{description}
\end{Desc}
