\section{winwidget.h File Reference}
\label{winwidget_8h}\index{winwidget.h@{winwidget.h}}
Component infrastructure for the window system. 

{\tt \#include $<$hpgraphics.h$>$}\par
\subsection*{Defines}
\begin{CompactItemize}
\item 
\#define {\bf WIN\_\-REPAINT\_\-EVENT}~7\label{winwidget_8h_a0}

\begin{CompactList}\small\item\em Event number for repaint events. \item\end{CompactList}\end{CompactItemize}
\subsection*{Typedefs}
\begin{CompactItemize}
\item 
typedef win\_\-widget {\bf win\_\-widget\_\-t}
\begin{CompactList}\small\item\em Data type for a widget. \item\end{CompactList}\end{CompactItemize}
\subsection*{Functions}
\begin{CompactItemize}
\item 
void {\bf win\_\-widget\_\-destroy} ({\bf win\_\-widget\_\-t} $\ast$widget)
\begin{CompactList}\small\item\em De-allocates a widget. \item\end{CompactList}\item 
void {\bf win\_\-add\_\-widget} ({\bf win\_\-widget\_\-t} $\ast$widget)
\begin{CompactList}\small\item\em Adds a widget to the window system, making it functional. \item\end{CompactList}\item 
void {\bf win\_\-remove\_\-widget} ({\bf win\_\-widget\_\-t} $\ast$widget)
\begin{CompactList}\small\item\em Removes a widget from the window system. \item\end{CompactList}\item 
void {\bf win\_\-focus\_\-next} (void)
\begin{CompactList}\small\item\em Moves focus to the next widget. \item\end{CompactList}\item 
void {\bf win\_\-focus\_\-prev} (void)
\begin{CompactList}\small\item\em Moves focus to the previous widget. \item\end{CompactList}\item 
int {\bf win\_\-widget\_\-has\_\-focus} ({\bf win\_\-widget\_\-t} $\ast$widget)
\begin{CompactList}\small\item\em Determines if a widget has input focus. \item\end{CompactList}\item 
void {\bf win\_\-widget\_\-set\_\-location} ({\bf win\_\-widget\_\-t} $\ast$widget, int x, int y)
\begin{CompactList}\small\item\em Sets the location of a widget on the screen. \item\end{CompactList}\item 
void {\bf win\_\-widget\_\-get\_\-location} ({\bf win\_\-widget\_\-t} $\ast$widget, int $\ast$x, int $\ast$y)
\begin{CompactList}\small\item\em Retrieves the location of a widget on the screen. \item\end{CompactList}\item 
void {\bf win\_\-widget\_\-set\_\-size} ({\bf win\_\-widget\_\-t} $\ast$widget, int x, int y)
\begin{CompactList}\small\item\em Sets the size of a widget on the screen. \item\end{CompactList}\item 
void {\bf win\_\-widget\_\-get\_\-size} ({\bf win\_\-widget\_\-t} $\ast$widget, int $\ast$w, int $\ast$h)
\begin{CompactList}\small\item\em Retrieves the size of a widget on the screen. \item\end{CompactList}\item 
void {\bf win\_\-widget\_\-pack} ({\bf win\_\-widget\_\-t} $\ast$widget)
\begin{CompactList}\small\item\em Sets the widget to its preferred size. \item\end{CompactList}\item 
void {\bf win\_\-widget\_\-set\_\-colors} ({\bf win\_\-widget\_\-t} $\ast$widget, unsigned char fgcolor, unsigned char bgcolor)
\begin{CompactList}\small\item\em Sets the colors of a widget. \item\end{CompactList}\item 
void {\bf win\_\-widget\_\-get\_\-colors} ({\bf win\_\-widget\_\-t} $\ast$widget, unsigned char $\ast$fgcolor, unsigned char $\ast$bgcolor)
\begin{CompactList}\small\item\em Retrieves the colors of a widget. \item\end{CompactList}\item 
void {\bf win\_\-widget\_\-set\_\-font} ({\bf win\_\-widget\_\-t} $\ast$widget, hpg\_\-font\_\-t $\ast$font)
\begin{CompactList}\small\item\em Sets the font of a widget. \item\end{CompactList}\item 
hpg\_\-font\_\-t $\ast$ {\bf win\_\-widget\_\-get\_\-font} ({\bf win\_\-widget\_\-t} $\ast$widget)
\begin{CompactList}\small\item\em Retrieves the font of a widget. \item\end{CompactList}\item 
void {\bf win\_\-widget\_\-set\_\-transparent} ({\bf win\_\-widget\_\-t} $\ast$widget, unsigned val)
\begin{CompactList}\small\item\em Sets the transparency of a widget. \item\end{CompactList}\item 
void {\bf win\_\-repaint} (void)
\begin{CompactList}\small\item\em Triggers a repaint of the screen. \item\end{CompactList}\end{CompactItemize}


\subsection{Detailed Description}
Component infrastructure for the window system. 

The bulk of the HPGCC window system consists of widgets. The functions and constants defined here provide the basis of the widget model.

Coming soon: this space will hold documentation on: common operations for all widgets (which don't exist yet - planned are functions to set size and location, focusability, and enable/disable); how to build a custom widgets, etc. Also, I will add a description of the surprising differences between HPGCC's window system and others (for example, there is no widget hierarchy; all widgets exist on their own). 

Definition in file {\bf winwidget.h}.

\subsection{Typedef Documentation}
\index{winwidget.h@{winwidget.h}!win_widget_t@{win\_\-widget\_\-t}}
\index{win_widget_t@{win\_\-widget\_\-t}!winwidget.h@{winwidget.h}}
\subsubsection{\setlength{\rightskip}{0pt plus 5cm}typedef struct win\_\-widget {\bf win\_\-widget\_\-t}}\label{winwidget_8h_a1}


Data type for a widget. 

This opaque type is used to identify individual widgets to the window system. It is returned from all functions that create widgets, and is later used to manage and/or destroy those widgets. 

Definition at line 65 of file winwidget.h.

\subsection{Function Documentation}
\index{winwidget.h@{winwidget.h}!win_add_widget@{win\_\-add\_\-widget}}
\index{win_add_widget@{win\_\-add\_\-widget}!winwidget.h@{winwidget.h}}
\subsubsection{\setlength{\rightskip}{0pt plus 5cm}void win\_\-add\_\-widget ({\bf win\_\-widget\_\-t} $\ast$ {\em widget})}\label{winwidget_8h_a3}


Adds a widget to the window system, making it functional. 

This is equivalent to other window and GUI frameworks' concepts of showing a widget or making it visible. The widget will appear immediately on the screen.

\begin{Desc}
\item[Parameters:]
\begin{description}
\item[{\em widget}]The widget to add. \end{description}
\end{Desc}
\index{winwidget.h@{winwidget.h}!win_focus_next@{win\_\-focus\_\-next}}
\index{win_focus_next@{win\_\-focus\_\-next}!winwidget.h@{winwidget.h}}
\subsubsection{\setlength{\rightskip}{0pt plus 5cm}void win\_\-focus\_\-next (void)}\label{winwidget_8h_a5}


Moves focus to the next widget. 

The focus will pass to the next focusable widget, in the order the widgets were added to the window system. If there is no other focusable widget, then this function has no effect.

Although this function is typically called from within a widget, it may also be called from the application. \index{winwidget.h@{winwidget.h}!win_focus_prev@{win\_\-focus\_\-prev}}
\index{win_focus_prev@{win\_\-focus\_\-prev}!winwidget.h@{winwidget.h}}
\subsubsection{\setlength{\rightskip}{0pt plus 5cm}void win\_\-focus\_\-prev (void)}\label{winwidget_8h_a6}


Moves focus to the previous widget. 

The focus will pass to the previous focusable widget, in the order the widgets were added to the window system. If there is no other focusable widget, then this function has no effect.

Although this function is typically called from within a widget, it may also be called from the application. \index{winwidget.h@{winwidget.h}!win_remove_widget@{win\_\-remove\_\-widget}}
\index{win_remove_widget@{win\_\-remove\_\-widget}!winwidget.h@{winwidget.h}}
\subsubsection{\setlength{\rightskip}{0pt plus 5cm}void win\_\-remove\_\-widget ({\bf win\_\-widget\_\-t} $\ast$ {\em widget})}\label{winwidget_8h_a4}


Removes a widget from the window system. 

This is equivalent to other window and GUI frameworks' concepts of hiding a widget. The widget will no longer receive events, and will therefore not appear on the screen.

\begin{Desc}
\item[Parameters:]
\begin{description}
\item[{\em widget}]The widget to remove. \end{description}
\end{Desc}
\index{winwidget.h@{winwidget.h}!win_repaint@{win\_\-repaint}}
\index{win_repaint@{win\_\-repaint}!winwidget.h@{winwidget.h}}
\subsubsection{\setlength{\rightskip}{0pt plus 5cm}void win\_\-repaint (void)}\label{winwidget_8h_a18}


Triggers a repaint of the screen. 

The repaint will occur at at some point during the dispatch of a coming idle event. In order to avoid unnecessary painting when multiple changes occur, several consecutive repaints are combined in a single paint operation by the window system. \index{winwidget.h@{winwidget.h}!win_widget_destroy@{win\_\-widget\_\-destroy}}
\index{win_widget_destroy@{win\_\-widget\_\-destroy}!winwidget.h@{winwidget.h}}
\subsubsection{\setlength{\rightskip}{0pt plus 5cm}void win\_\-widget\_\-destroy ({\bf win\_\-widget\_\-t} $\ast$ {\em widget})}\label{winwidget_8h_a2}


De-allocates a widget. 

All memory used by the given widget will be freed to the heap.

\begin{Desc}
\item[Warning:]Before destroying a widget, the application should ensure that the widget has been removed from the window system, and that any residual events have been delivered. It is often sufficient to wait for the next idle event or return non-zero from the event handler where the widget is destroyed.\end{Desc}
\begin{Desc}
\item[Parameters:]
\begin{description}
\item[{\em widget}]The widget to be destroyed. \end{description}
\end{Desc}
\index{winwidget.h@{winwidget.h}!win_widget_get_colors@{win\_\-widget\_\-get\_\-colors}}
\index{win_widget_get_colors@{win\_\-widget\_\-get\_\-colors}!winwidget.h@{winwidget.h}}
\subsubsection{\setlength{\rightskip}{0pt plus 5cm}void win\_\-widget\_\-get\_\-colors ({\bf win\_\-widget\_\-t} $\ast$ {\em widget}, unsigned char $\ast$ {\em fgcolor}, unsigned char $\ast$ {\em bgcolor})}\label{winwidget_8h_a14}


Retrieves the colors of a widget. 

\begin{Desc}
\item[Parameters:]
\begin{description}
\item[{\em widget}]The widget to operate on. \item[{\em fgcolor}]Pointer to storage for the foreground color. \item[{\em bgcolor}]Pointer to storage for the background color. \end{description}
\end{Desc}
\index{winwidget.h@{winwidget.h}!win_widget_get_font@{win\_\-widget\_\-get\_\-font}}
\index{win_widget_get_font@{win\_\-widget\_\-get\_\-font}!winwidget.h@{winwidget.h}}
\subsubsection{\setlength{\rightskip}{0pt plus 5cm}hpg\_\-font\_\-t$\ast$ win\_\-widget\_\-get\_\-font ({\bf win\_\-widget\_\-t} $\ast$ {\em widget})}\label{winwidget_8h_a16}


Retrieves the font of a widget. 

\begin{Desc}
\item[Parameters:]
\begin{description}
\item[{\em widget}]The widget to operate on. \end{description}
\end{Desc}
\begin{Desc}
\item[Returns:]The widget's current font. \end{Desc}
\index{winwidget.h@{winwidget.h}!win_widget_get_location@{win\_\-widget\_\-get\_\-location}}
\index{win_widget_get_location@{win\_\-widget\_\-get\_\-location}!winwidget.h@{winwidget.h}}
\subsubsection{\setlength{\rightskip}{0pt plus 5cm}void win\_\-widget\_\-get\_\-location ({\bf win\_\-widget\_\-t} $\ast$ {\em widget}, int $\ast$ {\em x}, int $\ast$ {\em y})}\label{winwidget_8h_a9}


Retrieves the location of a widget on the screen. 

\begin{Desc}
\item[Parameters:]
\begin{description}
\item[{\em widget}]The widget to operate on. \item[{\em x}]Pointer to storage for the x coordinate of the widget. \item[{\em y}]Pointer to storage for the y coordinate of the widget. \end{description}
\end{Desc}
\index{winwidget.h@{winwidget.h}!win_widget_get_size@{win\_\-widget\_\-get\_\-size}}
\index{win_widget_get_size@{win\_\-widget\_\-get\_\-size}!winwidget.h@{winwidget.h}}
\subsubsection{\setlength{\rightskip}{0pt plus 5cm}void win\_\-widget\_\-get\_\-size ({\bf win\_\-widget\_\-t} $\ast$ {\em widget}, int $\ast$ {\em w}, int $\ast$ {\em h})}\label{winwidget_8h_a11}


Retrieves the size of a widget on the screen. 

\begin{Desc}
\item[Parameters:]
\begin{description}
\item[{\em widget}]The widget to operate on. \item[{\em w}]Pointer to storage for the width of the widget in pixels. \item[{\em h}]Pointer to storage for the height of the widget in pixels. \end{description}
\end{Desc}
\index{winwidget.h@{winwidget.h}!win_widget_has_focus@{win\_\-widget\_\-has\_\-focus}}
\index{win_widget_has_focus@{win\_\-widget\_\-has\_\-focus}!winwidget.h@{winwidget.h}}
\subsubsection{\setlength{\rightskip}{0pt plus 5cm}int win\_\-widget\_\-has\_\-focus ({\bf win\_\-widget\_\-t} $\ast$ {\em widget})}\label{winwidget_8h_a7}


Determines if a widget has input focus. 

\begin{Desc}
\item[Returns:]Non-zero if the widget has focus; or zero if it does not. \end{Desc}
\index{winwidget.h@{winwidget.h}!win_widget_pack@{win\_\-widget\_\-pack}}
\index{win_widget_pack@{win\_\-widget\_\-pack}!winwidget.h@{winwidget.h}}
\subsubsection{\setlength{\rightskip}{0pt plus 5cm}void win\_\-widget\_\-pack ({\bf win\_\-widget\_\-t} $\ast$ {\em widget})}\label{winwidget_8h_a12}


Sets the widget to its preferred size. 

The widget will be resized to the size it reports as its \char`\"{}preferred\char`\"{} size. The preferred size of a widget is the size that the widget believed it needs in order to look presentable on the screen. The means of determining the preferred size are widget-specific.

\begin{Desc}
\item[Parameters:]
\begin{description}
\item[{\em widget}]The widget to resize.\end{description}
\end{Desc}
\begin{Desc}
\item[See also:]{\bf win\_\-widget\_\-set\_\-size}{\rm (p.\,\pageref{winwidget_8h_a10})} \end{Desc}
\index{winwidget.h@{winwidget.h}!win_widget_set_colors@{win\_\-widget\_\-set\_\-colors}}
\index{win_widget_set_colors@{win\_\-widget\_\-set\_\-colors}!winwidget.h@{winwidget.h}}
\subsubsection{\setlength{\rightskip}{0pt plus 5cm}void win\_\-widget\_\-set\_\-colors ({\bf win\_\-widget\_\-t} $\ast$ {\em widget}, unsigned char {\em fgcolor}, unsigned char {\em bgcolor})}\label{winwidget_8h_a13}


Sets the colors of a widget. 

The widget colors are used in a widget-specific way to determine how to draw the widget on the screen. Typically, the background color is used to fill the entire area taken by the widget, while the foreground color is used for drawing the widget's actual contents. However, some widgets may use these colors in other ways, or may ignore them entirely.

Setting colors will not change the widget's transparency. If the widget is transparent, the value of the background color is likely to be ignored until it becomes opaque.

\begin{Desc}
\item[Parameters:]
\begin{description}
\item[{\em widget}]The widget to operate on. \item[{\em fgcolor}]The new foreground color. \item[{\em bgcolor}]The new background color. \end{description}
\end{Desc}
\index{winwidget.h@{winwidget.h}!win_widget_set_font@{win\_\-widget\_\-set\_\-font}}
\index{win_widget_set_font@{win\_\-widget\_\-set\_\-font}!winwidget.h@{winwidget.h}}
\subsubsection{\setlength{\rightskip}{0pt plus 5cm}void win\_\-widget\_\-set\_\-font ({\bf win\_\-widget\_\-t} $\ast$ {\em widget}, hpg\_\-font\_\-t $\ast$ {\em font})}\label{winwidget_8h_a15}


Sets the font of a widget. 

After this call, the widget's font is changed to the given value. The font is used to draw text on the widget. The exact use of the font is widget-specific, and some widgets may use other fonts for rendering. However, the intent is that this font will be used for the widget's text.

Setting the font may affect the preferred size of the widget. The widget's size is not modified by this call. If the size should be updated, you must call {\bf win\_\-widget\_\-pack}{\rm (p.\,\pageref{winwidget_8h_a12})} or {\bf win\_\-widget\_\-set\_\-size}{\rm (p.\,\pageref{winwidget_8h_a10})} to do so.

\begin{Desc}
\item[Parameters:]
\begin{description}
\item[{\em widget}]The widget to operate on. \item[{\em font}]The new font. \end{description}
\end{Desc}
\index{winwidget.h@{winwidget.h}!win_widget_set_location@{win\_\-widget\_\-set\_\-location}}
\index{win_widget_set_location@{win\_\-widget\_\-set\_\-location}!winwidget.h@{winwidget.h}}
\subsubsection{\setlength{\rightskip}{0pt plus 5cm}void win\_\-widget\_\-set\_\-location ({\bf win\_\-widget\_\-t} $\ast$ {\em widget}, int {\em x}, int {\em y})}\label{winwidget_8h_a8}


Sets the location of a widget on the screen. 

The precise meaning of this information depends on the component. For some components, it may be completely ignored.

\begin{Desc}
\item[Parameters:]
\begin{description}
\item[{\em x}]The x coordinate for the left edge of the widget. \item[{\em y}]The y coordinate for the top edge of the widget. \end{description}
\end{Desc}
\index{winwidget.h@{winwidget.h}!win_widget_set_size@{win\_\-widget\_\-set\_\-size}}
\index{win_widget_set_size@{win\_\-widget\_\-set\_\-size}!winwidget.h@{winwidget.h}}
\subsubsection{\setlength{\rightskip}{0pt plus 5cm}void win\_\-widget\_\-set\_\-size ({\bf win\_\-widget\_\-t} $\ast$ {\em widget}, int {\em x}, int {\em y})}\label{winwidget_8h_a10}


Sets the size of a widget on the screen. 

The precise meaning of this information depends on the component. For some components, it may be completely ignored.

\begin{Desc}
\item[Parameters:]
\begin{description}
\item[{\em widget}]The widget to resize. \item[{\em w}]The width of the widget in pixels. \item[{\em h}]The height of the widget in pixels.\end{description}
\end{Desc}
\begin{Desc}
\item[See also:]{\bf win\_\-widget\_\-pack}{\rm (p.\,\pageref{winwidget_8h_a12})} \end{Desc}
\index{winwidget.h@{winwidget.h}!win_widget_set_transparent@{win\_\-widget\_\-set\_\-transparent}}
\index{win_widget_set_transparent@{win\_\-widget\_\-set\_\-transparent}!winwidget.h@{winwidget.h}}
\subsubsection{\setlength{\rightskip}{0pt plus 5cm}void win\_\-widget\_\-set\_\-transparent ({\bf win\_\-widget\_\-t} $\ast$ {\em widget}, unsigned {\em val})}\label{winwidget_8h_a17}


Sets the transparency of a widget. 

After this call, the widget should appear with the new transparency. It is intended that a transparent widget does not overwrite its background. Setting transparency can be used, for example, to preserve a background image that is drawn in light grays by a high-priority event handler for {\bf WIN\_\-REPAINT\_\-EVENT}{\rm (p.\,\pageref{winwidget_8h_a0})}.

\begin{Desc}
\item[Parameters:]
\begin{description}
\item[{\em widget}]The widget to operate on. \item[{\em val}]1 if the widget should be transparent; 0 otherwise. \end{description}
\end{Desc}
